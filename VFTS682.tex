\documentclass{aa}
\usepackage{graphicx}
\usepackage[varg]{txfonts}
\usepackage[%draft, 
colorlinks,citecolor=blue,linkcolor=blue,urlcolor=blue]{hyperref}
\usepackage{amsmath}
\usepackage[usenames]{xcolor}
\usepackage{comment}
\usepackage{multirow}
\usepackage{chngpage}
\usepackage{lscape}
\usepackage{url}
\newcommand{\todo}[1]{{\large $\blacksquare$~\textbf{\color{red}[#1]}}~$\blacksquare$}
\newcommand{\udef}{\stackrel{\mathrm{def}}{=}}
% for tree diagram
\usepackage{forest}
\usepackage{tikz-qtree}
\usetikzlibrary{shadows,trees}

%Selma's comments
\definecolor{Wildstrawberry}{rgb}{1.0, 0.26, 0.64}
\newcommand{\SdM}[1]{{\color{Wildstrawberry}\bf{#1}}}
\newcommand{\newtext}[1]{{\color{ForestGreen}\bf{#1}}}

\renewcommand{\labelitemii}{$\bullet$}
\newcommand{\kms}{{\,\mathrm{km\ s^{-1}}}}
\newcommand{\Msun}{{\mathrm{M}_\odot}}

\DeclareRobustCommand{\Eqref}[1]{Eq.~\ref{#1}}
\DeclareRobustCommand{\Figref}[1]{Fig.~\ref{#1}}
\DeclareRobustCommand{\Tabref}[1]{Table~\ref{#1}}
\DeclareRobustCommand{\Secref}[1]{Sec.~\ref{#1}}

% \defcitealias{tauris:98}{TT98}
% \defcitealias{ramirez-agudelo:15}{R-A15}

\interfootnotelinepenalty=10000    % brute-forces the footnote not to break over two pages

  
\begin{document}

\title{VFTS682: a confirmed dynamical ejection?}

\author{M.~Renzo\inst{1} \and \todo{TBD}% S.~E.~de~Mink\inst{1} \and
  % S.~Justham\inst{2,3} \and A.~de~Koter\inst{1} \todo{etc..order to be decided}
  .} 

\institute{{Astronomical Institute Anton Pannekoek, University of
    Amsterdam, 1098 XH Amsterdam, The Netherlands}
  % \and{School of Astronomy \& Space Science, University of the Chinese
  %   Academy of Sciences, Beijing 100012, China}
  % \and{National Astronomical Observatories, Chinese Academy of
  %   Sciences, Beijing 100012, China}
}
  
\offprints{M.~Renzo, \href{mailto:m.renzo@uva.nl}{m.renzo@uva.nl}}
\date{}
\abstract{}

\keywords{stars: kinematics, stars: runaways, stars: individual: VFTS682}
\maketitle{}

\section{Introduction}
\label{sec:intro}

How do stars form is one longstanding question in astrophysics
\cite{lada:03, zinnecker:07}. It is particularly difficult for massive stars, because these are intrinsically rare
\citep[e.g.,][]{salpeter:55,kroupa:01, schneider:18}, evolve fast, and
remain enshrouded in their parent cloud during the formation
process. Moreover, observations of young massive stars reveal a
complicated multipliticy structure which requires
explanation \citep[][]{kobulnicky:07,mason:09,sana:11,sana:12,kiminki:12,chini:12,kobulnicky:14,almeida:17,demarco:17}. Understanding massive star formation, possibly as a
function of metallicity, is a key question given the present and upcoming
transient survey \citep[e.g., LSST, BlackGem, LIGO/Virgo O3][]{} which
will reveal transients associated to massive stars
evolution and death.

The second data release (DR2) from the Gaia satellite
\cite[][]{gaia:16,brown:18} allows us to test
these hypothesis using one particular star, VFTS682. 
This star is a
very massive \citep[$M_\mathrm{ZAMS}\simeq150\,M_\odot$,][]{bestenlehner:11,schneider:18} WNh5 star in the 30 Doradus region
of the Large Magellanic Cloud (LMC), and it is presently observed at a
projected distance of $\sim$$29$\,pc from the nearest cluster of
massive stars R136 \citep[][]{bestenlehner:11}. Based on the extremely high mass of this star and
its present day apparent isolation, \cite{bestenlehner:11} proposed it
might be a candidate for isolated star formation, or a ``slow runaway'' ejected
from R136 in the past. This second option is also supported by the
N-body simulations of \cite{fujii:11, banerjee:12}.   Many other
very massive stars are present in the surroundings of R136, and a more
detailed analysis on the larger sample is desirable.


Massive stars can in principle be ejected from R136 as a consequence of dynamical
interactions \citep[][]{poveda:67,leonard:91, evans:10, fujii:11,
  allison:12, oh:16}, or by the disruption of a binary by the first
core-collapse supernova
\citep[][]{zwicky:57,blaauw:61,dedonder:97,eldridge:11,renzo:18}. However,
R136 has an estimated age of $\lesssim2$\,Myr \citep[][]{sabbi:12}, which is
shorter than the shortest stellar lifetime
\citep[$\sim$3\,Myr, e.g.,][]{zapartas:17}, so one would not expect the binary
disruption scenario to be relevant for this cluster.


In this study, we combine the radial velocity measurements from the
VFTS survey \citep[][]{evans:11} with the proper motion from Gaia DR2
to reconstruct the three-dimensional velocity of VFTS682, and test the
null hypothesis that this star was ejected from R136. \todo{check the
  following} Our results indicate that R136 is the likely origin of
this star, and therefore isolated star formation is \emph{not}
required to explain it. However, we find a mild discrepancy between the
apparent age of VFTS682, its kinematic age, and the age estimates for
R136. 

In \Secref{sec:sample}, we describe the data selection and validation
process for VFTS 682 and for stars in the surrounding used to
define the local reference frame. \Secref{sec:results} presents our main findings. We conclude
by discussing the implications for theories of
star formation, N-body interactions, and binary evolution in
\Secref{sec:discussion}.

\section{Gaia DR2 data selection}
\label{sec:sample}

VFTS682 is labeled in the Gaia DR2
catalog\footnote{\url{https://gea.esac.esa.int/archive/}} with the
source id 4657685637907503744. The star has a
\texttt{visibility\_period} = 17, which counts how many observations have
been used to reconstruct its astrometric solution \citep[][]{lindengren:18}. Its G-band
magnitude is 15.65, cf. the V-band magnitude of 16.08
\citep[][]{evans:11, bestenlehner:11}, and the reported
\texttt{astrometric\_excess\_noise} = 0. These values suggest that the Gaia
data for VFTS682 are trustworthy. However, the effective temperature
reported is one order of magnitude lower than what found by
\cite{bestenlehner:11}, and the best fit parallax of this star is
negative. We do not use the effective temperature of the star anywhere
in this study, and we attribute the unphysical value of the parallax
to the large distance to the LMC. Our main findings do not rely on the
parallax value reported in the Gaia catalog.

We retrieve for VFTS682 the position in right ascension (RA) and declination (DE)
in the IGCS frame \cite[][]{brown:18}, its
proper motion components ($\mu_\mathrm{RA}$, and $\mu_\mathrm{DE}$,
respectively). For the radial velocity of VFTS682 and of the 30 Doradus
region as a whole, we instead use the VFTS data
as quoted in \cite{bestenlehner:11}. \Tabref{tab:vfts682} lists the values adopted throughout
this work for each of these quantities.

\begin{table}[tbp]
  \centering
    \caption{Astrometric parameters for VFTS682. The peculiar radial
    velocity $\delta v_\mathrm{rad}$ is obtained as the difference
    between the average radial velocity of the 30 Doradus region
    ($270\pm10\kms$) minus the radial velocity measured from the HeII $\lambda4686$
    line for VFTS682 ($315\pm15\kms$).}

  \begin{tabular}[htbp]{l|c|c}
    Parameter & Value & Source\\ \hline\hline
    RA \hfill[degree] &  84.73 $\pm$  0.036 & \multirow{4}{*}{Gaia DR2}\\
    DE \hfill [degree] & -69.07 $\pm$  0.05  & \\
    $\mu_\mathrm{RA}$  \hfill[$\mathrm{mas\ yr^{-1}}$] & 1.84 $\pm$ 0.07 & \\
    $\mu_\mathrm{DE}$  \hfill[$\mathrm{mas\ yr^{-1}}$] & 0.78 $\pm$ 0.08& \\
    $\delta v_\mathrm{rad}$  \hfill[$\kms$] & -45 $\pm$ 25 & \cite{bestenlehner:11}\\
    \hline
  \end{tabular}
  \label{tab:vfts682}
\end{table}

To compare the astrometry of VFTS682 and derive its peculiar motion,
we then select data from the Gaia DR2 catalog for two regions: the
``surroundings'' of VFTS 682, and the ``R136 cluster''. The
surrounding region is defined by all the stars in a target of 10 arcminutes around
VFTS682 fulfilling the following criteria: we require a G-band
magnitude brighter than 17, to mimic the completeness threshold for
the VFTS survey, \texttt{visibility\_period} >= 5, and
\texttt{astrometric\_excess\_noise} <= 1. This selection yields
roughly 17800 stars, including the subset of stars which we consider to be
part of R136, discussed below. We further require  these stars to have finite proper
motion components and relative errors, reducing further the sample to
14856 stars. % A more strict sample selection for the surroundings can
% be found in \cite{lennon:18}.

The ``R136 cluster'' is effectively defined by taking all the stars
within 25 arcseconds from R136a, one of the most massive members of
the cluster itself \citep[][]{crowther:10}, requiring the same
``quality'' criteria applied above. Because of the magnitude cut
however, this would yield an incredibly small sample of 15 stars only
% which makes no sense at all
\todo{improve definition
  of stars from R136, describe accordingly -- check with Danny}

Throughout this study, we assume the same distance of $50$\,kpc to the star, and to
the 30 Doradus region as a whole, since the parallax for VFTS682
listed in the Gaia DR2 catalog is negative.

The data retrieved, and the ipython notebook used for the analysis
presented here will be made available at \todo{probably git repo on bitbucket?}. 

\section{The kinematics of VFTS682}
\label{sec:results}

\begin{figure*}[htbp]
  \centering
  \includegraphics[width=\textwidth]{./figures/main_plot}  
  \caption{position and projected relative velocity to
    R136. \todo{load pretty picture on background, check scale, clean
      foreground stars with parallax, add cone of uncertainty}}
  \label{fig:main}
\end{figure*}



\section{Summary and Discussion}
\label{sec:discussion}

\begin{itemize}
\item VFTS682 is a bona fide runaway with $v\sim60\,\mathrm{km\
    s^{-1}}$ thrown out from R136. Both its speed and the age of the
  cluster are consitent with a dynamical ejection.
  \item VFTS682 comes from R136 as was expected by
  \cite{bestenlehner:11, fujii:11, banerjee:12}, so it does not
  require isolated SFH to be explained
\item apparent age tension (connect to VFTS16 as well).
\item is R136 a single young cluster or a merger
\item estimate the influence of the gravitational potential of R136,
  what is its total mass and relaxation time?
\end{itemize}

Random notes: vsini<200 form \cite{schneider:18}, age $1\pm 0.2$\,Myr
from \cite{schneider:18}

\bibliographystyle{aa}
\bibliography{bibliography/vfts682}

% \begin{acknowledgements}
%  I am grateful to S.~Torres who helped me p
% \end{acknowledgements}
\end{document}




%%% Local Variables:
%%% mode: latex
%%% TeX-master: t
%%% End:
