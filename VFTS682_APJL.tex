\documentclass[apjl,twocolumn]{emulateapj}
\usepackage{graphicx}
\usepackage[varg]{txfonts}
%\usepackage[pagebackref,breaklinks,colorlinks,citecolor=blue]{hyperref}        
\usepackage[%draft, 
pagebackref,colorlinks,citecolor=blue,linkcolor=blue,urlcolor=blue]{hyperref}
\renewcommand*{\backref}[1]{[#1]}  % gives neat back references in the pdf
\usepackage{aas_macros}
%\usepackage{amsmath} % clashes with emulateapjj for some reaons
\usepackage[usenames,dvipsnames]{xcolor}
\usepackage{comment}
\usepackage{multirow}
%\usepackage{chngpage}
%\usepackage{lscape}
\usepackage{url}
\newcommand{\todo}[1]{{\large $\blacksquare$~\textbf{\color{red}[#1]}}~$\blacksquare$}
\newcommand{\udef}{\stackrel{\mathrm{def}}{=}}
% for tree diagram
%\usepackage{forest}
%\usepackage{tikz-qtree}
%\usetikzlibrary{shadows,trees}

%Selma's comments
\definecolor{Wildstrawberry}{rgb}{1.0, 0.26, 0.64}
%\newcommand{\SdM}[1]{{{\color{Sepia}{#1}}}}
\newcommand{\SdM}[1]{{{\color{brown}{#1}}}}
%\renewcommand{\SdM}[1]{{{{#1}}}}

\newcommand{\newtext}[1]{{\color{ForestGreen}\bf{#1}}}

\renewcommand{\labelitemii}{$\bullet$}
\newcommand{\kms}{{\,\mathrm{km\ s^{-1}}}}
\newcommand{\Msun}{{\,\mathrm{M}_\odot}}
\newcommand{\Lsun}{{\,\mathrm{L}_\odot}}
\newcommand{\masyr}{{\,\mathrm{mas}\,\mathrm{yr}^{-1}}}


\DeclareRobustCommand{\Eqref}[1]{Eq.~\ref{#1}}
\DeclareRobustCommand{\Figref}[1]{Fig.~\ref{#1}}
\DeclareRobustCommand{\Tabref}[1]{Table~\ref{#1}}
\DeclareRobustCommand{\Secref}[1]{Sec.~\ref{#1}}

\interfootnotelinepenalty=10000    % brute-forces the footnote not to break over two pages


\begin{document}

% \title{\emph{Gaia} DR2 identifies VFTS 682 as the most massive % , dynamically
%                                 % ejected %% to fit in one line
% runaway star known to date}
\title{Gaia's constraints on the  $\sim$$150\,M_\odot$  runaway star candidate VFTS682}

\author{M.~Renzo$^{1}$, S.~E.~de~Mink$^{1}$, D.~J.~Lennon$^{2}$,
  R.~P.~van~der~Marel$^{3,4}$, I.~Platais$^{4}$, J. Bestenlehner$^{5}$,
  V.~H\'enault-Brunet$^{6}$,  S.~Justham$^{1,7,8}$,  A.~de~Koter$^{1}$,
  N.~Langer$^{9}$, H.~Sana$^{10}$, F.~R.~Schneider$^{11}$, J.~S.~Vink$^{12}$}
\affil{$^{1}${Astronomical Institute Anton Pannekoek, University of
    Amsterdam, 1098 XH Amsterdam, The Netherlands} \\
  $^{2}$ {ESA, European Space Astronomy Centre, Apdo. de Correos 78,
    E-28691 Villanueva de la Ca\~nada, Madrid, Spain} \\
  $^{3}$ {Space Telescope Science Institute, 3700 San Martin Drive,
    Baltimore, MD 21218, USA}\\
  $^{4}$ {Center for Astrophysical Sciences, Department of Physics \& Astronomy, Johns Hopkins University, Baltimore, MD 21218, USA}\\
  $^{5}${Department of Physics and Astronomy, Hicks Building,
    Hounsfield Road, University of Sheffield, Sheffield S3 7RH, UK}\\
  $^{6}$ {National Research Council, Herzberg Astronomy \&
    Astrophysics, 5071 West Saanich Road, Victoria, BC, V9E 2E7,
    Canada}\\
  $^{7}$ {School of Astronomy \& Space Science, University of the Chinese
    Academy of Sciences, Beijing 100012, China}\\
  $^{8}$ {National Astronomical Observatories, Chinese Academy of
    Sciences, Beijing 100012, China}\\
  $^{9}$ {Argelander-Instit\"ut f\"ur Astronomie, Universit\"at Bonn,
    Auf dem H\"ugel 71, 53121, Bonn, Germany}\\
  $^{10}$ {Institute of Astronomy, KU Leuven, Celestijnenlaan 200 D, B-3001 Leuven, Belgium}\\
  $^{11}$ {Department of Physics, University of Oxford, Keble Road,
    Oxford OX1 3RH, UK} \\
  $^{12}$ {Armagh Observatory, College Hill, Armagh BT61 9DG, UK}\\
}

 \thanks{Corresponding author:  M.~Renzo, \href{mailto:m.renzo@uva.nl}{m.renzo@uva.nl}}
\date{}
\begin{abstract}
 
 How very massive stars form is still among the most intriguing open
 questions in stellar astrophysics.  VFTS 682 is among the most
 extreme massive stars known, with an inferred initial mass of about
 $150\,M_\odot$. It is located in 30 Doradus at a projected distance of 29\,pc from the central cluster R136.  The absence of other massive stars in its immediate vicinity lead to two intriguing possible hypotheses posed in earlier work. Either it formed in relative isolation through a new mode of star formation or it was ejected dynamically from the central cluster. 
 
Aiming to shine light on this debate, we investigate the kinematics of  VFTS 682 as obtained by \emph{Gaia} and a \emph{Hubble Space Telescope} proper motion campaign.    We derive a projected velocity relative of to the cluster of $\sim$$30\pm20\kms$.  The direction and magnitude of the proper motion are consistent with ejection from the central cluster considering its inferred age.  However, the error bars and spread in proper motions derived for other sources in the region are substantial, which makes us cautious to derive hard conclusions.   
 
If future data confirms the runaway nature, this would make the VFTS682 the most massive runaway star known to date. 
While we cannot prove this solidly from the current data, we do
consider this hypothesis plausible. The central cluster is known to
harbor several other stars more massive than $150\,M_\odot$ similar in
spectral type.  Moreover, the current record holders are O type stars
including VFTS 16 which provides direct evidence that R136 is indeed
capable of ejecting some of its most massive members. 
\end{abstract}

\keywords{stars: kinematics, stars: runaways, stars: individual: VFTS 682}
\maketitle{}

\section{Introduction}
\label{sec:intro}

How massive stars form is one of the major longstanding questions in astrophysics
\citep[e.g.,][]{zinnecker:07}. Improving our understanding of massive star formation, and its
possible dependence on environment and metallicity, is crucial for understanding the role massive stars play within their host galaxies, but also for understanding the transients that mark their death and the compact remnants they leave behind.
%
Obtaining clues from observations has been challenging, because massive stars are intrinsically rare, 
% \citep[e.g.,][]{salpeter:55,kroupa:01, schneider:18}, 
evolve fast, typically reside in dense groups and remain enshrouded in
their parent cloud during the entirety of their formation
process. Important progress has been made on the theoretical side,
\citep[e.g.][]{bate:09,kuiper:15,rosen:16}, but the simulations of this multi-scale and multi-physics problem are computationally very expensive and therefore remain challenging.  

In has been proposed that most, if not all, stars form in clusters \citep{lada:03}, where massive stars are thought to reside in the innermost cores. In this picture, field stars are primarily the result of the dissolution of dense groups.  However, a significant population of massive stars exists in relative isolation, far from dense clusters or OB associations and their origin remains matter of debate \citep{gvaramadze:12, lamb:16,ward:18}.   One hypothesis to explain the population of relatively isolated massive stars is that they formed in the field. The alternative hypothesis is that these massive stars were ejected from the clusters in which they formed.  Such ejections may result from dynamical interactions \citep[e.g.,][]{poveda:67} or from the disruption of binary systems at the death of the companion  star \citep[e.g.,][]{zwicky:57, blaauw:61, renzo:18}. 
 
% A contribution to the debate on whether or not massive
% stars form in relative isolation was presented by
% \cite{bestenlehner:11} and \cite{bressert:12}, who discussed the case of the very massive star
% VFTS682.

One of the most extreme examples that has been considered in this debate is the very massive star VFTS682  \citep[][]{bestenlehner:11, bressert:12}. This star is located in the field of the 30 Doradus region in the Large Magellanic Cloud (LMC) and was studied as part of the multi-epoch spectroscopic VLT-FLAMES Tarantula Survey \citep[VFTS,][]{evans:11}. It is a hydrogen-rich Wolf-Rayet star of spectral type WNh5. Spectral analysis and comparison with evolutionary models lead to an inferred present-day mass of $\sim$$140^{+30}_{-16}\,M_\odot$ corresponding to an initial mass of $\sim$$150^{+30}_{-17}\,M_\odot$
\citep{schneider:18}. 
%
%$137.8^{+27.5}_{-15.9}\,M_\odot$,
%
This makes VFTS682 one of the most massive stars known and one of the most extreme objects in the region.
%
From the spectral point of view, it is reminiscent of the very
massive stars % , a.k.a. ``monster stars'',
 in the core of the R136 cluster \citep{dekoter:97,crowther:10, crowther:16}. 
 %
 In particular, a remarkable similarity exist between the
spectrum of VFTS682 and R136a3 for which \citet{crowther:16} report a
current mass estimate of $180^{+30}_{-30}\Msun$. R136 hosts
at least two more very massive WN5h stars, R136a1 and R136a2, whose
estimated current masses are even higher. 

%at a projected offset of 119.4 arcseconds, corresponding to 

VFTS682 stands out by its relative isolation at a projected distance of 119.4 arcseconds, corresponding to 
$\sim$$29$\,pc, from  the star cluster R136. \citet{bestenlehner:11}
considered two possible explanation for the offset: either
the star formed in situ as an isolated massive star, % , but remark that this
                                % poses a challenge for starformation
                                % theories (They don't)
or it was ejected from  R136. N-body simulations % of dynamical interaction in star clusters
indicate that the ejection of very massive stars like VFTS682 is expected \citep[e.g.][]{fujii:11, banerjee:12}.  This is supported by the recent findings of other massive runaway stars in the region based on proper motion studies.   \citet{platais:15,platais:18} analyze multi-epoch \emph{Hubble Space Telescope} (HST) photometry and identify 10 stars, which appear to be ejected from R136.   \citet{lennon:18} investigate the kinematics of  isolated O-type stars in the region using the second \emph{Gaia} data release \cite[DR2,][]{gaia:16,brown:18} and show that the proper motion, postion and direction of the $\sim100\Msun$ star VFTS16 is consistent with a runaway origin from R136. 


%Testing the origin of VFTS682 is not only important for constraining isolated star formation; it should also help to improve understanding of early star-cluster dynamics.

In this paper we present an analysis of the new kinematical
constraints for VFTS682 provided by \emph{Gaia} DR2 and constraints
from HST proper data  by \citet{platais:18}.   We discuss the
implications for the hypothesis that VFTS682 is a slow runaway star
ejected from R136.
% which was not part of the sample of \citet{lennon:18} because of its  spectral type. 



%%% cut for length reasons: all is said below
% Our results indicate that VFTS682 is a runaway star, although with
% low statistical significance. The
% direction and magnitude of the velocity vector are consistent with
% dynamical ejection from R136. The age inferred from the spectral
% analysis \citep[from][]{schneider:18} is consistent with the travel
% time we calculate for this star. % Therefore, the hypothesis that VFTS
% % 682 is formed in relative isolation is rejected.
% If our results are confirmed by future astrometric data releases, VFTS
% 682, with an inferred present day mass well above a hundred solar
% masses, is the most massive runaway star known to date. 

\section{Observations}
\label{sec:sample}

\subsection{ Overview of VFTS682 from previous studies \label{data:vfts683}}

The star VFTS682, located at right ascension (RA)
05$^\mathrm{h}$38$^\mathrm{m}$55.510$^\mathrm{s}$  and declination
(DEC) \mbox{-69$^\mathrm{o}$04'26.72''} J2000 \citep[][% see also
% \Tabref{tab:vfts682} for the coordinates from \emph{Gaia} DR2
]{evans:11}
was originally classified as a young stellar object based on its
mid-infrared excess \citep{gruendl:09}. \citet{evans:11} reclassified the
object as Wolf-Rayet star of spectral type WNh5 using multi-epoch
spectra covering $\lambda$4000--7000 from the
VFTS survey. % The spectra available covered $\lambda$4000--7000 and
% were taken at multiple epochs.

Using the same dataset,
\citet{bestenlehner:11} excluded the presence of a close companion
with high confidence from the absence of radial velocity variations.
\citet{bestenlehner:11} % further analyzed the spectra % using the non-LTE
% % model atmosphere code CMFGEN \citep{hillier:98}
also derived stellar
parameters and surface abundances. They inferred an extinction
of $A_V=4.45\pm0.12$, leading to the high luminosity
$\log_{10}(L/L_\odot) =  6.5\pm0.2$. \citet{schneider:18} estimated
a present-day mass for VFTS682 of $137.8^{+27.5}_ {-15.9}\Msun$, an
apparent age of $1.0^{+0.2}_{-0.2}$\,Myr and an inferred initial mass
of $150.0^{+28.7}_{-17.4}\Msun$.% , using the Bayesian analysis tool BONNSAI
% \citep{schneider:17}, based on the evolutionary models from
% \citet{brott:11} and \cite{kohler:15}.
% These values place VFTS682 around the
% ``canonical upper limit'' of $\sim$$150\Msun$ by \citet{figer:05}.

\citet{bestenlehner:11} reported for VFTS682 a wind mass loss rate of
$10^{-4.1\pm0.2}\,M_\odot \ \mathrm{yr}^{-1}$, % inferred from
% spectral analysis assuming a smooth outflow, i.e.,
not accounting for
clumping. The present-day surface helium mass
fraction is $Y=0.45\, (0.49)$ from the spectral analysis of
\cite{bestenlehner:11} (\citealt{rubio-diez:17}). This value can be
used to constrain the age of the star and its past evolution.

% The star is very similar from the spectral point of view to R136a3, located inside the star cluster
% R136 \citep{crowther:10}  for which \citet{crowther:16} report a
% current mass estimate of $180^{+30}_{-30}\Msun$. The R136 cluster hosts
% at least two more very massive WN5h stars, R136a1 and R136a2, whose
% estimated current masses are even higher. However, VFTS682 stands
% out by its isolation at a projected offset of 29\,pc from the center
% of the cluster \citep{bestenlehner:11}. 

Further worth noticing is the variability of the
star. \citet{parker:93} reported B and V magnitudes roughly 0.5\,magnitude
lower than found by \citet{evans:11}.
\citet{bestenlehner:11} discuss the Optical Gravitational Lensing
Experiment (OGLE-III) light curves \citep{udalski:08} and show a
variability in the V-band at $\sim$10\% level on a timescale of years.
They commented that this is unusual for Wolf-Rayet stars and more reminiscent
of Luminous Blue Variable (LBV) stars. % \citep[e.g.,][]{langer:94}. 

\cite{bressert:12} note that VFTS682 shows a significant radial
velocity offset compared to the nebular lines from the gas surrounding
it, suggesting peculiar motion in the line of sight direction.
\citet{bestenlehner:11} quote a radial velocity measured from the
N{\footnotesize V} $\lambda4944$ of   $300\pm10\kms$. The average radial velocity of the 30 Doradus region is 
$270\pm10\kms$. Whether this measurement truly reflects a peculiar radial velocity is debated. The determination of radial velocities for emission-lines stars is difficult because of the presence of an optically thick, non-homogeneous, and variable wind obscuring the photosphere. The radial velocity errors quoted above are only statistical, and do not quantify the (possibly larger)
systematic errors.  \todo{Possibly mention and cite paper in prep by Paco and student on Xshooter data?  Possibly mention (or even use) on their measurement for the RV?}


For simplicity, throughout this study, we assume the same
distance of $50$\,kpc to the star \citep[][]{pietrzynski:13}, and to
the 30 Doradus region as a whole. We do not consider the error on
the distance determination ($\lesssim2\%$) when converting proper motions into
physical velocities.
%We implicitly assume that the use of a slightly different reference frame for $\delta v_\mathrm{rad}$ does not introduce significant errors. 

\subsection{HST astrometry for VFTS682}
% observations; the first epoch (GO-12499; P.I.: D.~J.~Lennon)
% in October 3-8 2011 and the second epoch (GO-13359) in
% October 6-11 2014, both using the F775W filter.
% The second epoch was designed to match the pointing and
% orientation of the first epoch as exactly as possible so
% that the displacement measurements could be purely differential.


For large distances ($\gtrsim50$\,kpc), HST astrometry can compete or even exceed \emph{Gaia} DR2 data in quality.

Recently, \citet{platais:18} presented HST astrometry and proper
motions in the 30 Doradus region (GO-12499; P.I.: D.~J.~Lennon). The
brightest stars (V$<$14) are missing from the catalogue because of
saturation, but the high extinction towards VFTS682 makes is faint
enough to be included. % in this catalogue.
The  proper motions are
listed in \Tabref{tab:vfts682}. 


The catalog from \citet{platais:18} lists the proper motion
components of VFTS682 relative to the field, providing a
measure of the motion of this star completely independent from
\emph{Gaia}. They found proper
motion components relative to the field
$\delta\mu_\mathrm{RA}^\mathrm{HST} = 0.01\pm0.13\,\mathrm{mas\
  yr^{-1}}$ and
$\delta\mu_\mathrm{DEC}^\mathrm{HST}=0.2\pm0.1\,\mathrm{mas\
  yr^{-1}}$, which added in quadrature give a relative proper motion of
$\delta \mu^{HST} =0.20 \pm 0.15\,\mathrm{mas\ yr^{-1}}$.
% \begin{equation}
%   \label{eq:pm_around_HST}
%   \delta \mu^{HST} = \sqrt{\left(\delta\mu_\mathrm{RA}^{HST}\right)^2+\left(\delta\mu_\mathrm{DEC}^{HST}\right)^2}
%   = 0.20 \pm 0.15\,\mathrm{mas\
%   yr^{-1}} \ \ .
% \end{equation}

These proper motion components can also be converted into projected
velocities assuming a distance of 50\,kpc, obtaining $\delta
v_\mathrm{RA}^{HST}=2\pm31\kms$ and $\delta
v_\mathrm{RA}^{HST}=47\pm24\kms$% , corresponding to a two-dimensional
% projected velocity of $47\pm24\kms$
.
Adding them in quadrature to the radial velocity
measured by \citet{bestenlehner:11} we obtain a three-dimensional velocity of
$v_\mathrm{pec}^{HST}=56 \pm 23  \kms$.
% \begin{equation}
%   \label{eq:speed_around_HST}
%   v_\mathrm{pec}^{HST} = \sqrt{\left(\delta v_\mathrm{RA}^{HST}\right)^2
%     +\left(\delta v_\mathrm{DEC}^{HST}\right)^2+\left(\delta
%       v_\mathrm{rad}^{VFTS}\right)^2} = 56 \pm 23
%   \kms \ .
% \end{equation}


\subsection{New data from \emph{Gaia} DR2  \label{data:gaia}}

\begin{table}
  \begin{center}
    \caption{Observed and derived parameters for VFTS682. We also list
    the average proper motion components of R136 derived by
    \cite{lennon:18} for reference.}
    \begin{tabular}{l|c|c}
      \hline
      \hline
      Parameter & Value & Source\\
      \hline
      RA \hfill[degree] &  \phantom{-}84.73 % $\pm$  0.03
                        & \multirow{2}{*}{\cite{evans:11}}\\[5pt]
      DEC \hfill [degree] & -69.07 % $\pm$  0.05
                        & \\[5pt]
      \hline
      $\delta\mu_\mathrm{RA}^{Gaia}$  \hfill[$\mathrm{mas\ yr^{-1}}$] & 0.10$\pm$0.07 & \multirow{2}{*}{\emph{Gaia} DR2}\\[5pt]
      $\delta\mu_\mathrm{DEC}^{Gaia}$  \hfill[$\mathrm{mas\ yr^{-1}}$] & 0.08$\pm$0.09 & \\[5pt]
      \hline
      $\delta\mu_\mathrm{RA}^{HST}$  \hfill[$\mathrm{mas\ yr^{-1}}$] & 0.01$\pm$0.13 & \multirow{2}{*}{\cite{platais:18}}\\[5pt]
      $\delta\mu_\mathrm{DEC}^{HST}$  \hfill[$\mathrm{mas\ yr^{-1}}$] &
                                                                        0.2$\pm$0.1 &
      \\[5pt]
      \hline
      $\delta v_\mathrm{rad}$  \hfill[$\kms$] & 30$\pm$20 & \cite{bestenlehner:11}\\
      \hline
      present day $M$ \hfill $[M_\odot]$ & $137.8^{+27.5}_
                                           {-15.9}$ &
                                                     \multirow{3}{*}{\cite{schneider:18}}\\
      $M_\mathrm{ZAMS}$\hfill $[M_\odot]$ & $150.0^{+28.7}_{-17.4}$ &
      \\
      age \hfill [Myr] & $1.0\pm0.2$ & \\
      \hline
      $\langle\mu_\mathrm{RA}\rangle$\hfill[$\mathrm{mas\ yr^{-1}}$] & $1.74\pm0.01$
                        &  \multirow{2}{*}{\cite{lennon:18}}\\[5pt]
      $\langle\mu_\mathrm{DEC}\rangle$\hfill[$\mathrm{mas\ yr^{-1}}$]
                & $0.70\pm0.02$ & \\
      \hline

    \end{tabular}
    \tablecomments
    {The error on the RA and DEC positions, are of order
      $\sim$$10^{-2}\,\mathrm{mas\ yr^{-1}}$ in \emph{Gaia}
      DR2. %REPETITION: %  We adopt the peculiar
      % radial velocity calculated using the  NV $\lambda4944$ line, which is
      % less sensitive to the wind velocity structure (see also \Secref{data:vfts683}).
    }
  \end{center}
  \label{tab:vfts682}
\end{table}


% \begin{table}
%   \begin{center}
%     \caption{Astrometric parameters for VFTS682. }
%     \begin{tabular}{l|c|c}
%       \hline
%       \hline
%       Parameter & Value & Source\\
%       \hline
%       RA \hfill[degree] &  \phantom{-}84.73 % $\pm$  0.03
%                         & \multirow{2}{*}{\cite{evans:11}}\\[5pt]
%       DEC \hfill [degree] & -69.07 % $\pm$  0.05
%                         & \\[5pt]
%       \hline
%       $\delta\mu_\mathrm{RA}^{Gaia}$  \hfill[$\mathrm{mas\ yr^{-1}}$] & 0.10$\pm$0.07 & \multirow{2}{*}{\emph{Gaia} DR2}\\[5pt]
%       $\delta\mu_\mathrm{DEC}^{Gaia}$  \hfill[$\mathrm{mas\ yr^{-1}}$] & 0.08$\pm$0.09 & \\[5pt]
%       \hline
%       $\delta\mu_\mathrm{RA}^{HST}$  \hfill[$\mathrm{mas\ yr^{-1}}$] & 0.01$\pm$0.13 & \multirow{2}{*}{\cite{platais:18}}\\[5pt]
%       $\delta\mu_\mathrm{DEC}^{HST}$  \hfill[$\mathrm{mas\ yr^{-1}}$] &
%                                                                         0.2$\pm$0.1 &
%       \\[5pt]
%       \hline
%       $\delta v_\mathrm{rad}$  \hfill[$\kms$] & 30$\pm$20 & \cite{bestenlehner:11}\\
%       \hline
%     \end{tabular}
%     \tablecomments
%     {The error on the RA and DEC positions, are of order
%       $\sim$$10^{-2}\,\mathrm{mas\ yr^{-1}}$ in \emph{Gaia}
%       DR2. %REPETITION: %  We adopt the peculiar
%       % radial velocity calculated using the  NV $\lambda4944$ line, which is
%       % less sensitive to the wind velocity structure (see also \Secref{data:vfts683}).
%     }
%   \end{center}
%   \label{tab:vfts682}
% \end{table}

The second \emph{Gaia} Data Release (DR2) % , which
% became available on 25 April 2018 and 
provides positions,
parallaxes, and proper motion components for more than a billion
sources \citep{brown:18}. VFTS682 is identified with the source id 4657685637907503744 in the \emph{Gaia} DR2
catalog\footnote{\url{https://vizier.u-strasbg.fr/viz-bin/VizieR-3?-source=I/345/gaia2}}. Its
G-band magnitude from \emph{Gaia}
% reports a G-band magnitude of
is 15.65. The star has a
\texttt{visibility\_period} = 17, which counts how many observations have
been used to reconstruct its astrometric solution
\citep[][]{lindengren:18}, and the reported
\texttt{astrometric\_excess\_noise} = 0. These values suggest that the \emph{Gaia}
data for VFTS682 are reliable. % However, the effective temperature
% reported in \emph{Gaia} DR2, based on the spectral fitting around the
% CaII triplet, is one order of magnitude lower than what found by
% \cite{bestenlehner:11}, and the best fit parallax of this star is
% negative \citep[see, e.g.,][]{hogg:18}. We do not use the effective temperature of the star anywhere
% in this study, and we attribute the unphysical value of the parallax
% to the large distance to the LMC.
% Our main findings do not rely on the
% parallax nor the effective temperature values reported in the \emph{Gaia} DR2
% catalog.

While \emph{Gaia} DR2 data can be correlated, this is
not an issue for VFTS682, and we treat the proper motion components ($\mu_\mathrm{RA}$, and $\mu_\mathrm{DEC}$,
respectively) of the star as uncorrelated for simplicity.
For the radial velocity of VFTS682 and of the 30 Doradus
region as a whole, we instead use the VFTS data
as quoted in \cite{bestenlehner:11}. \Tabref{tab:vfts682} lists the values adopted throughout
this work for each of these quantities.

We follow \citet{lennon:18} to define a local frame of reference and derive the peculiar velocity
of VFTS682 with respect to the cluster R136. Their selection (see
their Sec.~2.1) of 153
bright ($G<17$) stars around R136 with reliable astrometric data from
\emph{Gaia} DR2 yields $\langle\mu_\mathrm{RA}\rangle=1.74\pm0.01\,\mathrm{mas\
  yr^{-1}}$ and $\langle\mu_\mathrm{DEC}\rangle=0.70\pm0.02\,\mathrm{mas\ yr^{-1}}$
for the components of the mean motion of the region projected on the sky.

% We define a local standard frame of reference to derive the peculiar velocity
% of VFTS682 with respect to the cluster R136 by selecting from the \emph{Gaia} DR2 catalog following closely the approach of \cite{vandermarel:02,lennon:18}.
% We select all the stars in a target of 0.05 degrees around R136
% (NGC2070) fulfilling the following criteria. First, we require a G-band
% magnitude brighter than 17, corresponding roughly to the
% completeness level of the VFTS survey \citep[here we implicitly assume
% G$\sim$V,][]{evans:11}. Then we require \texttt{visibility\_period} $\geq$ 5,
% \texttt{astrometric\_excess\_noise} $< 1$, and the errors on the proper
% motion components to be smaller than 0.07\,$\mathrm{mas\
% yr^{-1}}$. At the distance to the LMC, $0.1\mathrm{mas\
% yr^{-1}}\simeq25\,\kms$ \citep[e.g.,][]{platais:18}. We
% remove obvious foreground stars by requiring the absolute value of
% the reported parallaxes smaller than $0.1\,\mathrm{mas\ yr^{-1}}$.
% At the distance to the
% LMC, $1\mathrm{mas\ yr^{-1}}\simeq250\,\kms$ \citep[e.g.,][]{lennon:18}, so the cut on the values
% of the proper motions removes stars that would have projected
% tangential velocities in excess of $\sim$500\,$\kms$, which are most
% likely to be foreground stars. We checked that the additional
% requirement of having parallaxes smaller than $2\,\mathrm{mas}$ does
% not reduces further our sample.
% This selection yields a sample of 66 stars.% ,

% We calculate the averaged proper motion components for the whole
% region using 
% \begin{equation}
%   \label{eq:mean}
%   \langle \mu_i\rangle = \frac{\sum_\mathrm{stars}\frac{1}{\Delta
%   \mu_i^2}\mu_i}{\sum_\mathrm{stars} \frac{1}{\Delta \mu_i^2}} \ \ , \
%   \ \Delta \langle \mu_i\rangle = \frac{\sqrt{N}}{\sum_\mathrm{stars}
%   \frac{1}{\Delta \mu_i^2}} \ \ ,
% \end{equation}
% where $i = \mathrm{RA}, \mathrm{DEC}$, and $\Delta \mu_i$ is the error
% on the proper motion component reported by \emph{Gaia}. The sums run over
% all the $N=66$ stars in our selected sample. We evaluate each proper motion
% component separately.


\section{The kinematics of VFTS682}
\label{sec:results}

\subsection{Is it a runaway star?}
\label{sec:runaway}
We first address the question of whether VFTS682 is a typical star
from the kinematic point of view, or whether it is a runaway star with
a significantly large peculiar velocity compared to its surrounding population. The former is what should
be expected if it formed in the relative isolation that we observe today.% , in relative
% isolation from other massive stars.

% Using the stars selected as described in \Secref{data:gaia} (a
% subset is shown in blue in \Figref{fig:main}), we find averaged proper motion components of
% $\langle\mu_\mathrm{RA}\rangle = 1.7240\pm0.0002\,\mathrm{mas\ yr^{-1}}$ and
% $\langle\mu_\mathrm{DEC}\rangle = 0.6986\pm0.0003\,\mathrm{mas\
% yr^{-1}}$. These values are
% in good agreement with what is found by \cite{lennon:18}. However, we
% emphasize their sensitivity to the choice of stars adopted to define
% the reference frame.
Subtracting the mean proper motion components given by
\citet{lennon:18} from the
proper motion of VFTS682 (see \Tabref{tab:vfts682}), we obtain the
components of proper motion of the star relative to the surrounding region
$\delta\mu_\mathrm{RA}^{Gaia} = 0.10 \pm 0.07\,\mathrm{mas\ yr^{-1}}$
and $\delta\mu_\mathrm{DEC}^{Gaia} = 0.08
\pm 0.09\,\mathrm{mas\ yr^{-1}}$. These components result in a
two-dimensional relative proper motion of $\delta \mu^{Gaia}=0.13\pm 0.09\,\mathrm{mas\
  yr^{-1}}$.

% \begin{equation}
%   \label{eq:pm_gaia_around}
%   \delta \mu^{Gaia} = \sqrt{\left(\delta\mu_\mathrm{RA}^{Gaia}\right)^2+\left(\delta\mu_\mathrm{DEC}^{Gaia}\right)^2}
%   = 0.13\pm 0.09\,\mathrm{mas\
%   yr^{-1}} \ \ .
% \end{equation}

% Figure \Figref{fig:pm_polar} shows the distribution in relative proper motion
% of the stars used in \cite{lennon:18} to define the frame of reference
% we also adopt here, cf.~the right panel in their Fig.~2. The
% proper motions are decomposed in the radial and tangential direction
% from R136, to highlight the likelihood of it being the origin of fast
% moving stars. The plus sign marks VFTS682, which sits at the edge of
% the proper motion distribution, but is not a clear outlier. Most
% relevant is the direction of the proper motion of VFTS682, which we
% discuss in \Secref{sec:r136_origin}.

% \begin{figure}%[htbp]
%   \centering
%   \includegraphics[width=0.5\textwidth]{figures/figure_polar_682-1.pdf}
%   \caption{Polar plot of the relative proper motion components for the stars
%     defining our reference frame. VFTS682 is indicated by the plus sign, and
%     other notable outliers \citep[see][]{lennon:18} are also labeled. Concentric dashed-lines denote relative proper motions of
%     0.1, 0.2, and 0.4$\,\mathrm{mas\ yr^{-1}}$. Positive angle
%     indicate a tangential component pointing in the counterclockwise
%     direction, with 0 degrees corresponding to proper motion pointing
%     radially outward from R136. The proper motion for VFTS682
%     corresponds to almost radial motion away from the cluster.
%   }
%   \label{fig:pm_polar}
% \end{figure}

The proper motion components can be converted into
the components of the relative transverse velocity $\delta v_\mathrm{RA}^{Gaia}=24\pm19\,\kms$,
$\delta v_\mathrm{DEC}^{Gaia}=20\pm23\,\kms$, assuming a distance of
50\,kpc. These can be combined obtaining a projected two-dimensional
velocity of $31\pm21\kms$. % (we do not account for the uncertainty in the distance
% estimate when propagating errors).
The radial velocity from
\cite{bestenlehner:11} then gives the third component along
the line of sight, % allowing us to calculate the three-dimensional
% peculiar speed of the star:
which added in quadrature to the transverse components results in a
three-dimensional velocity of $44 \pm 21\kms$.
% \begin{equation}
%   \label{eq:speed_around_Gaia}
%   v_\mathrm{pec}^{Gaia} = \sqrt{\left(\delta v_\mathrm{RA}^{Gaia}\right)^2
%     +\left(\delta v_\mathrm{DEC}^{Gaia}\right)^2+\left(\delta
%       v_\mathrm{rad}^{VFTS}\right)^2} = 44 \pm 21
%   \kms \ .
% \end{equation}

Therefore, two completely independent measures of the proper motion of
VFTS682 relative to the surrounding field, one from HST and one from
\emph{Gaia} DR2, yield values of the peculiar three-dimensional
velocity of VFTS682 % (\Eqref{eq:speed_around_HST} and
% \Eqref{eq:speed_around_Gaia}, respectively)
which would make it the most massive runaway star
known to date. However, the large errors on both the proper motion measures
require confirmation with future astrometric data. 

\subsection{Does it come from the R136 cluster?}
\label{sec:r136_origin}

% Assuming that the best estimate of the peculiar velocity of VFTS682
% are reliable, we now address the question of its likely origin.

Figure~\ref{fig:mu_dist} shows the distribution in proper motion relative to R136
of OB-type and Wolf-Rayet stars included both in the VFTS survey and
\emph{Gaia DR2} with reliable astrometric solutions. The bottom panel
shows the distribution in angles $\theta$ between the relative proper motion
direction and the direction from the core of R136 to the star. We
emphasize that our subset of stars with reliable proper motion
measurements is biased towards the fast moving objects, which results in a
distribution of angles mildly peaked at small angles.

VFTS682 is not an outlier in relative proper motion%  compared
% to the two O-type runaways (re-)identified by \cite{lennon:18} (see
% also below)
. However, the star was expected to be a ``slow runaway'' by \citet{bestenlehner:11} in
the dynamical ejection scenario. The green shade in
\Figref{fig:mu_dist} shows the range of relative proper motions
required to reach the present day location within
the uncertainties on the apparent age of the star, assuming it was
ejected from R136 very early in its life.

Because of the large
error bars, the angle between the relative proper motion is not very
constraining, but it is suggestive that the best value is close to zero.
Therefore, the relative proper motion from \emph{Gaia} DR2 are consistent with the hypothesis
of dynamical ejection.


\begin{figure}[htbp]
  \centering
  \includegraphics[width=0.5\textwidth]{figures/dist_mu_region.pdf}\\
  \includegraphics[width=0.5\textwidth]{figures/angle_dist}
  \caption{Top panel: distribution of OB-type and Wolf-Rayet stars in proper
    motion relative to R136. VFTS682 is not an outlier, but
    its relative proper motion matches the expected value if it were indeed
    ejected from R136 assuming an age of $1.0\pm0.2$\,Myr. The top axis shows the conversion to physical units
    assuming a distance of 50\,kpc. Bottom panel: \todo{redo with
      errorbars} distribution of
    angles between the relative proper motion direction and the radial
    direction to the star. The error bars for VFTS682 are large, but
    the best value is in agreement with the hypothesis of dynamical
    ejection. In both
    panels, the dark blue histograms contain 317 
    stars with error smaller than $0.1\,\mathrm{mas \
      yr^{-1}}\simeq25\,\mathrm{km\ s^{-1}}$ at 50\,kpc, the
    lighter blue histograms contain 36 stars with errors smaller than $0.05\,\mathrm{mas \
      yr^{-1}}$. }
  \label{fig:mu_dist}
\end{figure}


The red arrow in \Figref{fig:main} shows the direction of relative proper motion of
VFTS682 from \emph{Gaia} DR2, and the
yellow arrows illustrate the uncertainty. % show the possible
% range of directions within the uncertainties in the measured relative proper
% motion.
These arrows cross at the present-day location of VFTS682 and
are prolonged in the direction opposite to the motion to illustrate
the possible range of origins. % The yellow arrow shows the direction of
% the relative proper motion from HST (see \Eqref{eq:pm_around_HST}),
% but we omit to show the possible range of directions from HST data, since it is
% insufficiently constrained to be useful.

% Although the large uncertainties on the relative proper motion
% components results in a wide range of possible directions,we argue that the most likely origin of the star is R136.

The kinematic age of this star, assuming it originates from R136, is

\begin{equation}
  \label{eq:kin_age}
  \tau_\mathrm{kin} = \frac{d_\parallel}{\delta\mu^\mathrm{Gaia}} \simeq
  \frac{119.4\,\mathrm{arcsec}}{0.13\,\mathrm{mas\ yr^{-1}}} \simeq 0.9\pm\,0.6\, \mathrm{Myr} \ \ ,
\end{equation}
where $d_\parallel = 119.4\,\mathrm{arcsec}$ is the angular distance from VFTS682 to
the core of the cluster \citep[corresponding to $\sim$29\,pc at LMC distance,][]{bestenlehner:11}.
As in the rest of this study, we neglect for
simplicity the error on the distance estimates, because it is negligible compared to other uncertainties.
The kinematic age $\tau_\mathrm{kin}$ is compatible with a very early
ejection from the cluster, given is apparent age
of $1.0\pm 0.2$\,Myr \citep{schneider:18}. % , which
% corroborates the idea that the star is the result of a dynamical
% ejection very soon in the cluster evolution.

% the thing below is not relevant if it is a merger, better cut out
% for now, since we are too long anyway

%Note that the
% present-day surface helium abundance
% \citep[$Y\simeq0.5$,][]{bestenlehner:11, rubio-diez:17} puts a lower
% limit on the age of the star of $\sim$0.9\,Myr, corresponding to the time needed to
% synthesize this amount of helium in the models from \cite{kohler:15}.


\begin{figure}%[tbp]
  \centering
  \includegraphics[width=0.48\textwidth]{./figures/main_plot_good_gaia_only}  
  \caption{The red solid arrow indicates the proper motion of VFTS682
    relative to the region from \emph{Gaia} DR2, starting from the present day position of
    the star. The yellow arrows indicate the possible
    directions of projected motion within the \emph{Gaia} DR2 errors, and are extended
    backwards (dashed) to illustrate the uncertainty on the origin of the
    star. The length of the prolongations is proportional to the relative proper motion
    times the age of VFTS682 \citep[$1.0\pm0.2$\,Myr,][]{schneider:18}.
    % \todo{load color picture on background? Fig with HST data
    % available, but confusing...}
  }
  
  \label{fig:main}
\end{figure}


\section{Discussion}
\label{sec:discussion}

Based on our results, we tentatively claim that VFTS682 is the most massive
runaway known to date, with a peculiar three-dimensional speed of
$44\pm21\,\kms$. Due to the large error bars, this result will need
to be revisited with future astrometric data. % to be revisited as updated
% astrometric parameters from future \emph{Gaia} data releases become
% available.
If confirmed, it means that isolated star formation is
\emph{not} required to explain the isolation of VFTS682. Its proper motion suggests that it was ejected from the cluster R136
$0.9\pm0.6$\,Myr ago. Because of the exceptionally large mass
of this star, this raises the question of which stars must populate
the core of the cluster.

Dynamical ejections due to N-body interactions typically (although, not necessarily) eject the least
massive star among those interacting \cite[e.g.,][]{banerjee:12}. This means that, just
based on the kinematic properties of VFTS682, we would expect several
stars with initial masses larger than $\sim$$150\,M_\odot$ in the
cluster R136.
This is consistent with the detection
of extremely massive stars in the core of the
cluster. % The projected rotational equatorial
% velocity\footnote{However, the determination of the rotational
%   velocity for stars showing lines in emission is complicated by the optically thick wind screening the surface of the star, and should be
%   considered with caution.} of VFTS682
% reported by \cite{schneider:18} is $v\sin(i)<200\,\kms$, which is in
% line with the average rotation rate of massive stars in the region
% \citep[][]{ramirez-agudelo:15}. This suggests that VFTS682 (i) has not
% experienced rotationally induced chemically homogeneous evolution
% \citep[][]{maeder:00,demink:09}, and (ii) it has not
% accreted mass from or merged with a binary companion, nor will it, since the multi-epoch
% data of the VFTS survey rule out the presence of a close companion at the
% present day. Moreover,

The spectral type of VFTS682
\citep[WNh5,][]{bestenlehner:11} is the same as R136a1-a3, i.e.~the
three most massive stars detected in the core of the cluster%  by
% \cite{dekoter:97,crowther:10,crowther:16}
, with an astonishing similarity in particular with
the spectrum of R136a3. Therefore, the isolation of
VFTS682 makes it an ideal target to constrain the stellar physics of
stars with masses well above $\sim$$100\,M_\odot$ while avoiding
crowding issues. %  : its isolation makes
% it an easier target for observations compared to the similar stars
% present in the crowded core of R136.

\citet{banerjee:12} used N-body simulations of fully segregated
clusters with all massive stars in binaries to suggest that VFTS682
was ejected from R136. They
demonstrated that the cluster potential does not significantly change
the velocity of the star after the ejection. In their
model, they relied on (dynamically driven) stellar mergers to explain the high masses of
VFTS682 and the massive members of R136.%  While it is not possible to
% robustly exclude that VFTS682 is itself a merger, its spectrum does not show any of the obvious signatures
% like fast rotation, which however could have already slowed down
% because of the wind angular momentum losses during and after the
% merger.

To eject such a massive object, the cluster is
expected to have produced
a large number of massive runaways. Indeed, several %relatively
isolated massive stars are observed in the region, some with known
large radial velocities and/or proper motion. % (see
% \Figref{fig:pm_polar} and Sana et al., in prep.).
A comprehensive study of the kinematic
properties of all the massive stars surrounding R136 might shed light
on whether some can be unequivocally identified as merger products. It
is also possible that the star or binary that caused the ejection of
VFTS682 might have been ejected in the opposite direction, and is also
isolated at present day. If the ejection was caused by an interaction
with a binary, however, it is likely that the binary scattered in the
opposite direction will experience further dynamical interactions on
its way, modifying its trajectory and making it difficult to find.  %absence of proof would not undermine our conclusions 

The similarities between VFTS682 and the WNh5 stars in the core of
R136 are also in agreement with the ``bully binary'' model of
\cite{fujii:11}. Based on their numerical results, they suggested that
early in the evolution of a cluster, dynamical interactions form an extremely
massive binary, which then tightens its orbit by ejecting other stars passing
by. Interpreting our results for VFTS682 through the lens of their simulations
suggests the presence of a close binary with total mass
$M_1+M_2\gtrsim 300\,M_\odot$ in the core of the cluster. Such bully
binary could be R145 according to \cite{fujii:11}, and it might be an
ideal observational candidate for a dynamically formed progenitor system of
a binary black-hole, provided that stars this massive can avoid a
pair-instability supernova \cite[e.g.,][]{rakavy:67} at LMC
metallicity \citep[see also][]{langer:07}. Similarly, the final fate of VFTS682 could be either a
pair-instability supernova without compact remnant formation, or
possibly direct collapse to a black hole above the $2^\mathrm{nd}$
mass gap. The amount of mass loss of these stars will determine their final core
mass and thus their final fate.

The kinematic age of VFTS682 puts an
upper limit to the timescale to form the ``bully binary'' in
R136. The cluster must have been at the very beginning of its
evolution, given the age estimate of $\lesssim 2$\,Myr
\citep[][]{crowther:10,sabbi:12} and the kinematic age of VFTS682. If the
cluster is indeed younger than the shortest stellar lifetime
\citep[$\sim$3\,Myr, e.g.,][]{brott:11, zapartas:17}, then the alternative
explanation for ejection of VFTS682 from the disruption of a binary
by a core-collapse event is excluded since the region is too young for stars
to have experienced core-collapse already.

The variability of VFTS682, reminiscent of LBV stars, suggests
that VFTS682 (and therefore its analogs in the core of R136) might
experience enhanced mass loss episodes in LBV eruptions. \citet{smith:15} made the highly
debated\footnote{See, e.g., \cite{humphreys:16, davidson:16, smith:16}.}
claim that LBV stars are typically isolated form O-type stars. The fact that VFTS682 is a dynamically
ejected runaway which might evolve into an LBV star suggests that
N-body interactions also play a role in explaining the apparent
isolation of at least some LBV stars. 


\citet{lennon:18} carried out a study similar to ours on the fast
moving O-type stars
in the region, and found two massive runaway stars
($\sim$$90\,M_\odot$) in the 30 Doradus region. One of them (VFTS 16)
was previously known as a runaway star from its line of sight velocity
\citep[][]{evans:10}. \citet{lennon:18} also concluded that VFTS16 is 
the result of a dynamical ejection from the R136 cluster, while the
origin of the other star (VFTS 72) is less clear given its direction
of motion. The value of $\tau_\mathrm{kin}\simeq0.9$\,Myr we find for
VFTS682 (see \Secref{sec:r136_origin}) is smaller than the
corresponding value for VFTS16: \cite{lennon:18} inferred a kinematic
age of $\sim$1.5\,Myr, possibly in tension with the apparent age of that star. This means that the more
massive VFTS682 was ejected later than VFTS16 from the same cluster.

The numerical simulations from \cite{oh:16} suggest that dynamical
interaction eject the majority of the stars during or shortly after the cluster
core-collapse. The large number of isolated massive stars around it
suggest that R136 has already evolved past the
time of maximum stellar density. This might have implications for the
question of whether the cluster formed via a monolithic collapse, or
as a (potentially ongoing) merger of several sub-structures \citep[e.g.,][]{sabbi:12}.

\citet{oh:16} also showed that the mass and velocity distribution of the ejected star depends on the cluster initial conditions
(whether it is segregated, its primordial binary fraction and initial period
distribution of the binary population), therefore studies on
the population of isolated massive stars in the surroundings of R136
might shed light on its initial stellar population and dynamical
state. 

VFTS682 is potentially the most massive runaway known to date, and its ejection
from the cluster R136 likely implies that it is only the ``tip of the
iceberg'' of possibly extremely massive runaways in the
region. Studies of this population, enabled by recent HST and \emph{Gaia} observations will put constraints on the evolution
of these extreme stars, together with the formation and evolution of
the central cluster itself.



% \bibliographystyle{aa}
% \bibliography{bibliography/vfts682}
\begin{thebibliography}{}
\makeatletter
\relax
\def\mn@urlcharsother{\let\do\@makeother \do\$\do\&\do\#\do\^\do\_\do\%\do\~}
\def\mn@doi{\begingroup\mn@urlcharsother \@ifnextchar [ {\mn@doi@}
  {\mn@doi@[]}}
\def\mn@doi@[#1]#2{\def\@tempa{#1}\ifx\@tempa\@empty \href
  {http://dx.doi.org/#2} {doi:#2}\else \href {http://dx.doi.org/#2} {#1}\fi
  \endgroup}
\def\mn@eprint#1#2{\mn@eprint@#1:#2::\@nil}
\def\mn@eprint@arXiv#1{\href {http://arxiv.org/abs/#1} {{\tt arXiv:#1}}}
\def\mn@eprint@dblp#1{\href {http://dblp.uni-trier.de/rec/bibtex/#1.xml}
  {dblp:#1}}
\def\mn@eprint@#1:#2:#3:#4\@nil{\def\@tempa {#1}\def\@tempb {#2}\def\@tempc
  {#3}\ifx \@tempc \@empty \let \@tempc \@tempb \let \@tempb \@tempa \fi \ifx
  \@tempb \@empty \def\@tempb {arXiv}\fi \@ifundefined
  {mn@eprint@\@tempb}{\@tempb:\@tempc}{\expandafter \expandafter \csname
  mn@eprint@\@tempb\endcsname \expandafter{\@tempc}}}

\bibitem[\protect\citeauthoryear{{Banerjee}, {Kroupa}  \& {Oh}}{{Banerjee}
  et~al.}{2012}]{banerjee:12}
{Banerjee} S.,  {Kroupa} P.,   {Oh} S.,  2012, \mn@doi [\apj]
  {10.1088/0004-637X/746/1/15}, \href
  {http://adsabs.harvard.edu/abs/2012ApJ...746...15B} {746, 15}

\bibitem[\protect\citeauthoryear{{Bate}}{{Bate}}{2009}]{bate:09}
{Bate} M.~R.,  2009, \mn@doi [\mnras] {10.1111/j.1365-2966.2008.14106.x}, \href
  {http://adsabs.harvard.edu/abs/2009MNRAS.392..590B} {392, 590}

\bibitem[\protect\citeauthoryear{{Bestenlehner} et~al.,}{{Bestenlehner}
  et~al.}{2011}]{bestenlehner:11}
{Bestenlehner} J.~M.,  et~al., 2011, \mn@doi [\aap]
  {10.1051/0004-6361/201117043}, \href
  {http://adsabs.harvard.edu/abs/2011A\%26A...530L..14B} {530, L14}

\bibitem[\protect\citeauthoryear{{Blaauw}}{{Blaauw}}{1961}]{blaauw:61}
{Blaauw} A.,  1961, \bain, \href
  {http://adsabs.harvard.edu/abs/1961BAN....15..265B} {15, 265}

\bibitem[\protect\citeauthoryear{{Bressert} et~al.,}{{Bressert}
  et~al.}{2012}]{bressert:12}
{Bressert} E.,  et~al., 2012, \mn@doi [\aap] {10.1051/0004-6361/201117247},
  \href {http://adsabs.harvard.edu/abs/2012A\%26A...542A..49B} {542, A49}

\bibitem[\protect\citeauthoryear{{Brott} et~al.,}{{Brott}
  et~al.}{2011}]{brott:11}
{Brott} I.,  et~al., 2011, \mn@doi [\aap] {10.1051/0004-6361/201016113}, \href
  {http://adsabs.harvard.edu/abs/2011A\%26A...530A.115B} {530, A115}

\bibitem[\protect\citeauthoryear{{Crowther}, {Schnurr}, {Hirschi}, {Yusof},
  {Parker}, {Goodwin}  \& {Kassim}}{{Crowther} et~al.}{2010}]{crowther:10}
{Crowther} P.~A.,  {Schnurr} O.,  {Hirschi} R.,  {Yusof} N.,  {Parker} R.~J.,
  {Goodwin} S.~P.,   {Kassim} H.~A.,  2010, \mn@doi [\mnras]
  {10.1111/j.1365-2966.2010.17167.x}, \href
  {http://adsabs.harvard.edu/abs/2010MNRAS.408..731C} {408, 731}

\bibitem[\protect\citeauthoryear{{Crowther} et~al.,}{{Crowther}
  et~al.}{2016}]{crowther:16}
{Crowther} P.~A.,  et~al., 2016, \mn@doi [\mnras] {10.1093/mnras/stw273}, \href
  {http://adsabs.harvard.edu/abs/2016MNRAS.458..624C} {458, 624}

\bibitem[\protect\citeauthoryear{{Davidson}, {Humphreys}  \& {Weis}}{{Davidson}
  et~al.}{2016}]{davidson:16}
{Davidson} K.,  {Humphreys} R.~M.,   {Weis} K.,  2016, arXiv:1608.02007, \href
  {http://adsabs.harvard.edu/abs/2016arXiv160802007D} {}

\bibitem[\protect\citeauthoryear{{de Koter}, {Heap}  \& {Hubeny}}{{de Koter}
  et~al.}{1997}]{dekoter:97}
{de Koter} A.,  {Heap} S.~R.,   {Hubeny} I.,  1997, \mn@doi [\apj]
  {10.1086/303736}, \href {http://adsabs.harvard.edu/abs/1997ApJ...477..792D}
  {477, 792}

  
\bibitem[\protect\citeauthoryear{{Evans} et~al.,}{{Evans}
  et~al.}{2010}]{evans:10}
{Evans} C.~J.,  et~al., 2010, \mn@doi [\apjl] {10.1088/2041-8205/715/2/L74},
  \href {http://adsabs.harvard.edu/abs/2010ApJ...715L..74E} {715, L74}

\bibitem[\protect\citeauthoryear{{Evans} et~al.,}{{Evans}
  et~al.}{2011}]{evans:11}
{Evans} C.~J.,  et~al., 2011, \mn@doi [\aap] {10.1051/0004-6361/201116782},
  \href {http://adsabs.harvard.edu/abs/2011A\%26A...530A.108E} {530, A108}

\bibitem[\protect\citeauthoryear{{Fujii} \& {Portegies Zwart}}{{Fujii} \&
  {Portegies Zwart}}{2011}]{fujii:11}
{Fujii} M.~S.,  {Portegies Zwart} S.,  2011, \mn@doi [Science]
  {10.1126/science.1211927}, \href
  {http://adsabs.harvard.edu/abs/2011Sci...334.1380F} {334, 1380}

\bibitem[\protect\citeauthoryear{{Gaia Collaboration} et~al.,}{{Gaia
  Collaboration} et~al.}{2016}]{gaia:16}
{Gaia Collaboration} et~al., 2016, \mn@doi [\aap]
  {10.1051/0004-6361/201629272}, \href
  {http://adsabs.harvard.edu/abs/2016A\%26A...595A...1G} {595, A1}

\bibitem[\protect\citeauthoryear{{Gaia Collaboration}, {Brown}, {Vallenari},
  {Prusti}, {de Bruijne}, {Babusiaux}  \& {Bailer-Jones}}{{Gaia Collaboration}
  et~al.}{2018}]{brown:18}
{Gaia Collaboration} {Brown} A.~G.~A.,  {Vallenari} A.,  {Prusti} T.,  {de
  Bruijne} J.~H.~J.,  {Babusiaux} C.,   {Bailer-Jones} C.~A.~L.,  2018,
  ArXiv:1804.09365, \href {http://adsabs.harvard.edu/abs/2018arXiv180409365G}
  {}

\bibitem[\protect\citeauthoryear{{Gruendl} \& {Chu}}{{Gruendl} \&
  {Chu}}{2009}]{gruendl:09}
{Gruendl} R.~A.,  {Chu} Y.-H.,  2009, \mn@doi [\apjs]
  {10.1088/0067-0049/184/1/172}, \href
  {http://adsabs.harvard.edu/abs/2009ApJS..184..172G} {184, 172}

\bibitem[\protect\citeauthoryear{{Gvaramadze}, {Weidner}, {Kroupa}  \&
  {Pflamm-Altenburg}}{{Gvaramadze} et~al.}{2012}]{gvaramadze:12}
{Gvaramadze} V.~V.,  {Weidner} C.,  {Kroupa} P.,   {Pflamm-Altenburg} J.,
  2012, \mn@doi [\mnras] {10.1111/j.1365-2966.2012.21452.x}, \href
  {http://adsabs.harvard.edu/abs/2012MNRAS.424.3037G} {424, 3037}

% \bibitem[\protect\citeauthoryear{{Hogg}}{{Hogg}}{2018}]{hogg:18}
% {Hogg} D.~W.,  2018, ArXiv:1804.07766, \href
%   {http://adsabs.harvard.edu/abs/2018arXiv180407766H} {}

\bibitem[\protect\citeauthoryear{{Humphreys}, {Weis}, {Davidson}  \&
  {Gordon}}{{Humphreys} et~al.}{2016}]{humphreys:16}
{Humphreys} R.~M.,  {Weis} K.,  {Davidson} K.,   {Gordon} M.~S.,  2016, \mn@doi
  [\apj] {10.3847/0004-637X/825/1/64}, \href
  {http://adsabs.harvard.edu/abs/2016ApJ...825...64H} {825, 64}

\bibitem[\protect\citeauthoryear{{K{\"o}hler} et~al.,}{{K{\"o}hler}
  et~al.}{2015}]{kohler:15}
{K{\"o}hler} K.,  et~al., 2015, \mn@doi [\aap] {10.1051/0004-6361/201424356},
  \href {http://adsabs.harvard.edu/abs/2015A\%26A...573A..71K} {573, A71}

\bibitem[\protect\citeauthoryear{{Kuiper}, {Yorke}  \& {Turner}}{{Kuiper}
  et~al.}{2015}]{kuiper:15}
{Kuiper} R.,  {Yorke} H.~W.,   {Turner} N.~J.,  2015, \mn@doi [\apj]
  {10.1088/0004-637X/800/2/86}, \href
  {http://adsabs.harvard.edu/abs/2015ApJ...800...86K} {800, 86}

\bibitem[\protect\citeauthoryear{{Lada} \& {Lada}}{{Lada} \&
  {Lada}}{2003}]{lada:03}
{Lada} C.~J.,  {Lada} E.~A.,  2003, \mn@doi [\araa]
  {10.1146/annurev.astro.41.011802.094844}, \href
  {http://adsabs.harvard.edu/abs/2003ARA\%26A..41...57L} {41, 57}

\bibitem[\protect\citeauthoryear{{Lamb}, {Oey}, {Segura-Cox}, {Graus},
  {Kiminki}, {Golden-Marx}  \& {Parker}}{{Lamb} et~al.}{2016}]{lamb:16}
{Lamb} J.~B.,  {Oey} M.~S.,  {Segura-Cox} D.~M.,  {Graus} A.~S.,  {Kiminki}
  D.~C.,  {Golden-Marx} J.~B.,   {Parker} J.~W.,  2016, \mn@doi [\apj]
  {10.3847/0004-637X/817/2/113}, \href
  {http://adsabs.harvard.edu/abs/2016ApJ...817..113L} {817, 113}

\bibitem[\protect\citeauthoryear{{Langer}, {Norman}, {de Koter}, {Vink},
  {Cantiello}  \& {Yoon}}{{Langer} et~al.}{2007}]{langer:07}
{Langer} N.,  {Norman} C.~A.,  {de Koter} A.,  {Vink} J.~S.,  {Cantiello} M.,
  {Yoon} S.-C.,  2007, \mn@doi [\aap] {10.1051/0004-6361:20078482}, \href
  {http://adsabs.harvard.edu/abs/2007A\%26A...475L..19L} {475, L19}

\bibitem[\protect\citeauthoryear{{Lennon} et~al.,}{{Lennon}
  et~al.}{2018}]{lennon:18}
{Lennon} D.~J.,  et~al., 2018, ArXiv:1805.08277, \href
  {http://adsabs.harvard.edu/abs/2018arXiv180508277L} {}

\bibitem[\protect\citeauthoryear{{Lindegren} et~al.,}{{Lindegren}
  et~al.}{2018}]{lindengren:18}
{Lindegren} L.,  et~al., 2018, ArXiv:1804.09366, \href
  {http://adsabs.harvard.edu/abs/2018arXiv180409366L} {}

% \bibitem[\protect\citeauthoryear{{Maeder} \& {Meynet}}{{Maeder} \&
%   {Meynet}}{2000}]{maeder:00}
% {Maeder} A.,  {Meynet} G.,  2000, \aap, \href
%   {http://adsabs.harvard.edu/abs/2000A\%26A...361..159M} {361, 159}

\bibitem[\protect\citeauthoryear{{Oh} \& {Kroupa}}{{Oh} \&
  {Kroupa}}{2016}]{oh:16}
{Oh} S.,  {Kroupa} P.,  2016, \mn@doi [\aap] {10.1051/0004-6361/201628233},
  \href {http://adsabs.harvard.edu/abs/2016A\%26A...590A.107O} {590, A107}

\bibitem[\protect\citeauthoryear{{Parker}}{{Parker}}{1993}]{parker:93}
{Parker} J.~W.,  1993, \mn@doi [\aj] {10.1086/116661}, \href
  {http://adsabs.harvard.edu/abs/1993AJ....106..560P} {106, 560}

\bibitem[\protect\citeauthoryear{{Pietrzy{\'n}ski} et~al.,}{{Pietrzy{\'n}ski}
  et~al.}{2013}]{pietrzynski:13}
{Pietrzy{\'n}ski} G.,  et~al., 2013, \mn@doi [\nat] {10.1038/nature11878},
  \href {http://adsabs.harvard.edu/abs/2013Natur.495...76P} {495, 76}

\bibitem[\protect\citeauthoryear{{Platais}, {van der Marel}, {Lennon},
  {Anderson}, {Bellini}, {Sabbi}, {Sana}  \& {Bedin}}{{Platais}
  et~al.}{2015}]{platais:15}
{Platais} I.,  {van der Marel} R.~P.,  {Lennon} D.~J.,  {Anderson} J.,
  {Bellini} A.,  {Sabbi} E.,  {Sana} H.,   {Bedin} L.~R.,  2015, \mn@doi [\aj]
  {10.1088/0004-6256/150/3/89}, \href
  {http://adsabs.harvard.edu/abs/2015AJ....150...89P} {150, 89}

\bibitem[\protect\citeauthoryear{{Platais} et~al.,}{{Platais}
  et~al.}{2018}]{platais:18}
{Platais} I.,  et~al., 2018, ArXiv:1804.08678, \href
  {http://adsabs.harvard.edu/abs/2018arXiv180408678P} {}

\bibitem[\protect\citeauthoryear{{Poveda}, {Ruiz}  \& {Allen}}{{Poveda}
  et~al.}{1967}]{poveda:67}
{Poveda} A.,  {Ruiz} J.,   {Allen} C.,  1967, Boletin de los Observatorios
  Tonantzintla y Tacubaya, \href
  {http://adsabs.harvard.edu/abs/1967BOTT....4...86P} {4, 86}

\bibitem[\protect\citeauthoryear{{Rakavy} \& {Shaviv}}{{Rakavy} \&
  {Shaviv}}{1967}]{rakavy:67}
{Rakavy} G.,  {Shaviv} G.,  1967, \mn@doi [\apj] {10.1086/149204}, \href
  {http://adsabs.harvard.edu/abs/1967ApJ...148..803R} {148, 803}

% \bibitem[\protect\citeauthoryear{{Ram{\'{\i}}rez-Agudelo}
%   et~al.,}{{Ram{\'{\i}}rez-Agudelo} et~al.}{2015}]{ramirez-agudelo:15}
% {Ram{\'{\i}}rez-Agudelo} O.~H.,  et~al., 2015, \mn@doi [\aap]
%   {10.1051/0004-6361/201425424}, \href
%   {http://adsabs.harvard.edu/abs/2015A\%26A...580A..92R} {580, A92}

\bibitem[\protect\citeauthoryear{{Renzo} et~al.,}{{Renzo}
  et~al.}{2018}]{renzo:18}
{Renzo} M.,  et~al., 2018, ArXiv:1804.09164, \href
  {http://adsabs.harvard.edu/abs/2018arXiv180409164R} {}

% \bibitem[\protect\citeauthoryear{{Robitaille} \& {Bressert}}{{Robitaille} \&
%   {Bressert}}{2012}]{robitaille:12}
% {Robitaille} T.,  {Bressert} E.,  2012, {APLpy: Astronomical Plotting Library
%   in Python}, Astrophysics Source Code Library (\mn@eprint {ascl} {1208.017})

\bibitem[\protect\citeauthoryear{{Rosen}, {Krumholz}, {McKee}  \&
  {Klein}}{{Rosen} et~al.}{2016}]{rosen:16}
{Rosen} A.~L.,  {Krumholz} M.~R.,  {McKee} C.~F.,   {Klein} R.~I.,  2016,
  \mn@doi [\mnras] {10.1093/mnras/stw2153}, \href
  {http://adsabs.harvard.edu/abs/2016MNRAS.463.2553R} {463, 2553}

\bibitem[\protect\citeauthoryear{{Rubio-D{\'{\i}}ez}, {Najarro},
  {Garc{\'{\i}}a}  \& {Sundqvist}}{{Rubio-D{\'{\i}}ez}
  et~al.}{2017}]{rubio-diez:17}
{Rubio-D{\'{\i}}ez} M.~M.,  {Najarro} F.,  {Garc{\'{\i}}a} M.,   {Sundqvist}
  J.~O.,  2017, in {Eldridge} J.~J.,  {Bray} J.~C.,  {McClelland} L.~A.~S.,
  {Xiao} L.,  eds,  IAU Symposium Vol. 329, The Lives and Death-Throes of
  Massive Stars. pp 131--135, \mn@doi{10.1017/S1743921317002447}

\bibitem[\protect\citeauthoryear{{Sabbi} et~al.,}{{Sabbi}
  et~al.}{2012}]{sabbi:12}
{Sabbi} E.,  et~al., 2012, \mn@doi [\apjl] {10.1088/2041-8205/754/2/L37}, \href
  {http://adsabs.harvard.edu/abs/2012ApJ...754L..37S} {754, L37}

\bibitem[\protect\citeauthoryear{{Sana}, {Ram{\'{\i}}rez-Tannus}, {de Koter},
  {Kaper}, {Tramper}  \& {Bik}}{{Sana} et~al.}{2017}]{sana:17}
{Sana} H.,  {Ram{\'{\i}}rez-Tannus} M.~C.,  {de Koter} A.,  {Kaper} L.,
  {Tramper} F.,   {Bik} A.,  2017, \mn@doi [\aap]
  {10.1051/0004-6361/201630087}, \href
  {http://adsabs.harvard.edu/abs/2017A\%26A...599L...9S} {599, L9}

% \bibitem[\protect\citeauthoryear{{Schneider}, {Castro}, {Fossati}, {Langer}  \&
%   {de Koter}}{{Schneider} et~al.}{2017}]{schneider:17}
% {Schneider} F.~R.~N.,  {Castro} N.,  {Fossati} L.,  {Langer} N.,   {de Koter}
%   A.,  2017, \mn@doi [\aap] {10.1051/0004-6361/201628409}, \href
%   {http://adsabs.harvard.edu/abs/2017A\%26A...598A..60S} {598, A60}

\bibitem[\protect\citeauthoryear{{Schneider} et~al.,}{{Schneider}
  et~al.}{2018}]{schneider:18}
{Schneider} F.~R.~N.,  et~al., 2018, \mn@doi [Science]
  {10.1126/science.aan0106}, \href
  {http://adsabs.harvard.edu/abs/2018Sci...359...69S} {359, 69}

\bibitem[\protect\citeauthoryear{{Smith}}{{Smith}}{2016}]{smith:16}
{Smith} N.,  2016, \mn@doi [\mnras] {10.1093/mnras/stw1533}, \href
  {http://adsabs.harvard.edu/abs/2016MNRAS.461.3353S} {461, 3353}

\bibitem[\protect\citeauthoryear{{Smith} \& {Tombleson}}{{Smith} \&
  {Tombleson}}{2015}]{smith:15}
{Smith} N.,  {Tombleson} R.,  2015, \mn@doi [\mnras] {10.1093/mnras/stu2430},
  \href {http://adsabs.harvard.edu/abs/2015MNRAS.447..598S} {447, 598}

\bibitem[\protect\citeauthoryear{{Udalski} et~al.,}{{Udalski}
  et~al.}{2008}]{udalski:08}
{Udalski} A.,  et~al., 2008, \actaa, \href
  {http://adsabs.harvard.edu/abs/2008AcA....58..329U} {58, 329}

\bibitem[\protect\citeauthoryear{{Ward} \& {Kruijssen}}{{Ward} \&
  {Kruijssen}}{2018}]{ward:18}
{Ward} J.~L.,  {Kruijssen} J.~M.~D.,  2018, \mn@doi [\mnras]
  {10.1093/mnras/sty117}, \href
  {http://adsabs.harvard.edu/abs/2018MNRAS.475.5659W} {475, 5659}

\bibitem[\protect\citeauthoryear{{Zapartas} et~al.,}{{Zapartas}
  et~al.}{2017}]{zapartas:17}
{Zapartas} E.,  et~al., 2017, \mn@doi [\aap] {10.1051/0004-6361/201629685},
  \href {http://adsabs.harvard.edu/abs/2017A\%26A...601A..29Z} {601, A29}

\bibitem[\protect\citeauthoryear{{Zinnecker} \& {Yorke}}{{Zinnecker} \&
  {Yorke}}{2007}]{zinnecker:07}
{Zinnecker} H.,  {Yorke} H.~W.,  2007, \mn@doi [\araa]
  {10.1146/annurev.astro.44.051905.092549}, \href
  {http://adsabs.harvard.edu/abs/2007ARA\%26A..45..481Z} {45, 481}

\bibitem[\protect\citeauthoryear{{Zwicky}}{{Zwicky}}{1957}]{zwicky:57}
{Zwicky} F.,  1957, \zap, \href
  {http://adsabs.harvard.edu/abs/1957ZA.....44...64Z} {44, 64}


% \bibitem[\protect\citeauthoryear{{de Mink}, {Cantiello}, {Langer}, {Pols},
%   {Brott}  \& {Yoon}}{{de Mink} et~al.}{2009}]{demink:09}
% {de Mink} S.~E.,  {Cantiello} M.,  {Langer} N.,  {Pols} O.~R.,  {Brott} I.,
%   {Yoon} S.-C.,  2009, \mn@doi [\aap] {10.1051/0004-6361/200811439}, \href
%   {http://adsabs.harvard.edu/abs/2009A\%26A...497..243D} {497, 243}

\makeatother
\end{thebibliography}

% \begin{acknowledgements}
%   \small
%   We are grateful to S.~Torres, M.~C.~Ramirez-Tannus,
%   and C.~J.~Evans for help and discussions. SdM has received funding under the European Unions Horizon 2020 research and innovation programme from the European Research Council (ERC) (Grant agreement No. 715063). VHB acknowledges support from the NRC-Canada Plaskett Fellowship. This work has made use of data from the ESA space mission \emph{Gaia} (\url{http://www.cosmos.esa.int/gaia}), processed by the \emph{Gaia} Data Processing and Analysis Consortium (DPAC, \url{http://www.cosmos.esa.int/web/gaia/dpac/consortium}). Funding for the DPAC has been provided by national institutions, in particular the institutions participating in the \emph{Gaia} Multilateral Agreement. % The background image of \Figref{fig:main} is based on observations
%   % made with ESO Telescopes at the La Silla Observatory under programme
%   % ID 076.C-0888, processed and released by the ESO VOS/ADP group.
%   % This research made use of APLpy, an open-source plotting package for Python \citep[][]{robitaille:12}.
% \end{acknowledgements}

\end{document}




%%% Local Variables:
%%% mode: latex
%%% TeX-master: t
%%% End:
